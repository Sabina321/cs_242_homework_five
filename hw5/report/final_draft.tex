\documentclass{article}
\usepackage{amsmath}
\usepackage{verbatim}
\usepackage[margin=1in]{geometry}
\setlength\parindent{0pt}
\usepackage{color}
\usepackage{graphicx}
\usepackage{hyperref}
% \usepackage{subcaption}
%\usepackage{tabularx}
%\usepackage{float}

\begin{document}


\begin{center}
\Huge{\textsc{Homework Five Proposal}} 

\Large\textsc{CMPS242 - Fall 2015}\\

\large{Andrei Ignat  \hfill Sabina Tomkins \hfill Guangyu Wang} 
\end{center}

\section*{Data Exploration}
The first thing we wanted to look at was what the relative importance of different features are. We started this by looking at the correlation between different features. Here is what we saw: 

\begin{tabular}{|l|r|r|r|r|r|r|}
\hline
 &  Survived &    Pclass &       Age &     SibSp &     Parch &      Fare \\
\hline
Survived &  1.000000 & -0.323533 & -0.043385 &  0.085915 &  0.133933 &  0.266229 \\
\hline
Pclass   & -0.323533 &  1.000000 & -0.286081 & -0.039552 & -0.021019 & -0.573531 \\
\hline
Age      & -0.043385 & -0.286081 &  1.000000 & -0.142746 & -0.200112 &  0.093249 \\
\hline
SibSp    &  0.085915 & -0.039552 & -0.142746 &  1.000000 &  0.425241 &  0.358262 \\
\hline
Parch    &  0.133933 & -0.021019 & -0.200112 &  0.425241 &  1.000000 &  0.330360 \\
\hline
Fare     &  0.266229 & -0.573531 &  0.093249 &  0.358262 &  0.330360 &  1.000000 \\
\hline
\end{tabular}\\


From this table we can see that if we want to be predicting survival, some especially correlated attributes are fare and Pclass. Both of these things have to do with socio-economic status. \\

This prompts us to look at the percentage of each class which survived, which results in the following:


\begin{tabular}{|l|r|r|r|r|r|r|}
\hline
&  Class One &    Class Two &       Class Three\\
\hline
Average Survival &  .63 & .47 & .24\\
\hline

\end{tabular}

Fare:

Pclass:

Fare & Pclass:

Fare & Pclass & Gender


[ 0.78290812  0.81702493  0.76732997  0.72384262  0.79258112]

Fare & Pclass & Age


[ 0.59390631  0.69139755  0.68916206  0.804376    0.68170617]

Fare & Pclass & Embarked


[ 0.54053681  0.68715123  0.6117014   0.73344071  0.70122227]


["Pclass","Fare","Gender","Age"]


[ 0.78290812  0.81702493  0.76732997  0.72384262  0.79258112]
0.77673735022926149

["Pclass","Fare","Gender","Embarked"]
[ 0.78290812  0.81702493  0.76732997  0.72384262  0.79258112]
0.77673735022926149

All five: 0.77673735022926149


C=500
[ 0.71620556  0.83109993  0.77678818  0.75942139  0.82119062]
0.78094113770932438

In the case of the Linear SVM. Adding in age, and embarked gives no performance improvement. 

C=50
0.77673735022926149

Logistic Regression

Tried from 0-1000 in step sizes of 100, and saw no difference. .784 constant. 

\bibliography{proposal_biblio}
\bibliographystyle{plain}

\end{document}
